\documentclass[russian]{article}
\usepackage[utf8]{inputenc}
\usepackage[T1]{fontenc}
\usepackage[russian]{babel}
\usepackage{amsmath}
\usepackage{graphicx}
\usepackage{fancyhdr}
\usepackage[inline]{enumitem}
\usepackage{xcolor}
\pagestyle{fancy}
\fancyhf{}
\renewcommand{\headrulewidth}{0pt}
\setlength{\headheight}{40pt} 

\begin{document}

\title{\textbf{ПОДБОР ПРОТОКОЛОВ ИНТЕРНЕТ ВЕЩЕЙ В КАЧЕСТВЕ ОСНОВНЫХ ПРИ ПЕРЕДАЧИ ИНФОРМАЦИИ С ДАТЧИКОВ СТАНКОВ ЧПУ}}

\author{Хлебников А.~А.\textsuperscript{1}, Кудасов С.~В.\textsuperscript{2}, Курнасов Е.~В.\textsuperscript{3}}

\maketitle
\thispagestyle{fancy}

1. (viruszold@gmail.com) 2. (lulu@gmail.com)
\begin{abstract}
В данной работе проводится анализ популярных протоколов <<Интернет вещей>> исходя из требований 
надежности, качества и применимости к задаче обеспечения съема информации с мобильных датчиков станков
ЧПУ посредствам сетей передачи данных WiFi. Также проводится полевая проверка на более чем N датчиков в 
различных условиях работы.
\end{abstract}

\section*{Введение}

В связи с широким распростарением принципов <<Интернет вещей>>, стала появляться необходимость 
не просто взаимодействовать компонентам между собой, но и обеспечивать необходимую надежность, 
качество и быстроту передачи информации. В совокупности с применяемыми физическими средствами передачи, большую роль
играет и программные протоколы передачи, а их большое количество и разнообразие приводит к сложному и 
не однозначному анализу \cite{IoT_Protocols_Research}.


\section*{Протоколы}

Разобьем, для удобства, необходимые нам протоколы по назначению:
\begin{enumerate*}[label={\alph*)},font={\color{red!50!black}\bfseries}]
\item непосредственный, непрерывная передача информации 
\item системный, управления.
 \end{enumerate*}
И выясним, какие нам подходят для решения нашей задачи.
 
\subsection*{MQTT\cite{MQTT}}

...

\subsection*{CoAP\cite{CoAP}}

...

\subsection*{HTTP2}

...

\subsection*{DDS\cite{DDS}}

...

\subsection*{XMPP\cite{XMPP}}

...

\subsection*{Результат}

...

\section*{Условия проверки}

...


\section*{Критерии оценки}

...

\section*{Испытательный стенд}

\subsection*{Схема}
\subsection*{Устройства}
\subsection*{Условия}

\section*{Проведение испытаний}

...

\section*{Результаты}

...

\section*{Заключение}

...

%http://www.tssonline.ru/articles2/reviews/analiticheskiy-obzor-protokolov-interneta-veschey
\begin{thebibliography}{1}
\expandafter\ifx\csname url\endcsname\relax
  \def\url#1{\texttt{#1}}\fi
\expandafter\ifx\csname urlprefix\endcsname\relax\def\urlprefix{URL }\fi
\providecommand{\bibinfo}[2]{#2}
\providecommand{\eprint}[2][]{\url{#2}}

\bibitem{cite1}
\bibinfo{author}{Califano, A.}, \bibinfo{author}{Butte, A.~J.},
  \bibinfo{author}{Friend, S.}, \bibinfo{author}{Ideker, T.} \&
  \bibinfo{author}{Schadt, E.}
\newblock \bibinfo{title}{{Leveraging models of cell regulation and GWAS data
  in integrative network-based association studies}}.
\newblock \emph{\bibinfo{journal}{Nature Genetics}}
  \textbf{\bibinfo{volume}{44}}, \bibinfo{pages}{841--847}
  (\bibinfo{year}{2012}).

\bibitem{cite2}
\bibinfo{author}{Wang, R.} \emph{et~al.}
\newblock \bibinfo{title}{{PRIDE Inspector: a tool to visualize and validate MS
  proteomics data.}}
\newblock \emph{\bibinfo{journal}{Nature Biotechnology}}
  \textbf{\bibinfo{volume}{30}}, \bibinfo{pages}{135--137}
  (\bibinfo{year}{2012}).

\bibitem{IoT_Protocols_Research}
\bibinfo{author}{Фам~В.~Д.}, 
  \bibinfo{author}{Юльчиева~Л.~О.},
  \bibinfo{author}{Киричек~Р.~В}
\newblock \bibinfo{title}{{ИССЛЕДОВАНИЕ ПРОТОКОЛОВ ВЗАИМОДЕЙСТВИЯ ИНТЕРНЕТА ВЕЩЕЙ}}.
\newblock \emph{\bibinfo{journal}{Информационные технологии в телекоммуникации}}
  \textbf{\bibinfo{volume}{Том 4 №1}}, \bibinfo{pages}{55--67}
  (\bibinfo{year}{2006}).

\bibitem{DDS}
\newblock \bibinfo{title}{{OMG/ DDS v1.4 – the DDS specification}}.
\newblock \emph{\bibinfo{document}{Object Management Group}}
  (\bibinfo{year}{2015}).
  
\bibitem{XMPP}
\newblock \bibinfo{title}{{ITU-T/ Extensible Messaging and Presence Protocol (XMPP): Core}}.
\newblock \emph{\bibinfo{document}{RFC-3920}}
  (\bibinfo{year}{2004}).
  
\bibitem{CoAP}
\newblock \bibinfo{title}{{ITU-T/ The Constrained Application Protocol (CoAP)}}.
\newblock \emph{\bibinfo{document}{RFC 7252 – Proposed Standard}}
  (\bibinfo{year}{2014}).
 
\bibitem{MQTT}
\newblock \bibinfo{title}{{IBM/ MQTT V3.1 Protocol Specification}}.
\newblock \emph{\bibinfo{document}{International Business Machines Corporation Eurotech}}
  (\bibinfo{year}{2015}).
 
\bibitem{STOMP}
\newblock \bibinfo{title}{{STOMP Protocol Specification, Version 1.2}}.
\newblock \emph{\bibinfo{document}{licensed under the Creative Commons Attribution v2.5 license}}
  (\bibinfo{year}{2012}).
  
\end{thebibliography}

\end{document}